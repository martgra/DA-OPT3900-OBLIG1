\documentclass[12pt]{article}
\usepackage[utf8]{inputenc}
\usepackage[english,norsk]{babel}
\usepackage[left=1in, right=1in, top=1in, bottom=1in]{geometry}
\usepackage{fancyhdr}
\usepackage{apacite}
\usepackage{titling}
\usepackage{lipsum}



\pretitle{\begin{center}\Huge\bfseries}
\posttitle{\par\end{center}\vskip 0.5em}
\preauthor{\begin{center}\Large\ttfamily}
\postauthor{\end{center}}
\predate{\par\large\centering}
\postdate{\par}
\pagestyle{fancy}
\fancyfoot{}
\fancyfoot[R]{\thepage}
\renewcommand{\headrulewidth}{0pt}
\lhead{DA-OPT 3900: PROSJEKTRAPPORT FOR OBLIGATORISK OPPGAVE 1}
\rhead{}
\linespread{1.25}


\title{DA-OPT 3900:\\Prosjektrapport for obligatorisk oppgave 1}
\author{Amer Sisic, Thor-Stian Follstad og Martin Gran}
\date{\today}
\renewcommand*\contentsname{Innholdsfortegnelse}

\begin{document}
% setter starten p{\aa} seksjoner
\setcounter{section}{1}

\maketitle
\thispagestyle{myheadings}
\linespread{1.00}
\vspace{\fill}
\begin{abstract}
				This project report describes the implementation of together four different algorithms of the "Travelling salesman problem" (TSP).
				First a symetrical complete graph is implemented with random lengths of 1-10 between a random number of cities. Then a route is 
				chosen on random and the cost is calculated. The next solution is to choose routes at random and updating the best route by the 
				lowest cost. Third a greedy method is implemented. In the implementation the next city in the route is chosen by selecting
				the path with the lowest value. Fourth a greedy itterative method is chosen. The method works on a complete route, then switches
				cities at random and updates the route if the total cost is improved. The fourth method works on routes generated by the previous
				methods. Last the different solutions are run 100 times at the same graph. The best result and the mean result of the methods are then compared.
\\ \\ \\ \\ \\ \\ \\ \\ \\
\end{abstract}
\vspace{\fill}
\pagebreak
\thispagestyle{empty}%
\tableofcontents
\pagebreak

\addcontentsline{toc}{section}{1 Innledning}
\indent
				Denne prosjektrapporten beskriver fire forskjellige implementasjoner av algoritmer for {\aa} l{\o}se "the Travelling Salesman problem" (TSP).
				L{\o}sningen og rapporten er utarbeidet av Amer Sisic, Thor-Stian Follstad og Martin Gran. L{\o}sningsforslaget implementerer fire forskjellige algoritmer
				forskjellige algoritmer for {\aa} finne den beste l{\o}sningen av problemet. De fire algoritmene er tilfeldig metode (random metode), tilfeldig itterativ metode (random itterativ method), gr{\aa}dig 
				metode (greedy method) og gr{\aa}dig itterativ metode (greedy itterativ method). De implementerte algoritmene kj{\o}res deretter 100 ganger p{\aa} en identisk
				fullstendig, symetrisk graf som representerer byene med avstander. Deretter sammenlignes resultatene de forskjellige algoritmene produserer. F{\o}rst sammenlignes
				det beste resultatet hver enkelt algoritme klarte {\aa} produsere. Deretter sammenliknes det gjennomsnittlige resultatet hver enkelt algoritme produserer. \\\\
				Rapporten vil f{\o}rst dr{\o}fte den teoretiske bakgrunnen for TSP. Deretter vil rapporten peke p{\aa} avveininger gjort i forhold til implementering av algoritmene.
				Tilslutt vil oppgaven dr{\o}fte resultatene med bakgrunn i resultat i form av tid og kostnad. \cite{a} 
				Some text \( \circ \) and more and a centered dot and more.
\begin{enumerate}
\item hello
\begin{itemize}
\item hello
\end{itemize}
\end{enumerate}
				

%\pagebreak
%\section*{First bit}
%If you want to ramp your text straight onto the title page, start the text at 
%something that does not cause a page break, like a section.  Here's a handy 
%place to introduce some of your woofy conventions, like quotes in equations.

\section{Teoretisk bakgrunn}
\subsection{Travelling Salesman problemet}
				Opprinnelsen til "Travelling Salesman problem" kan spores tilbake til 1800-tallet \cite{b}. 
				Den nevnes blant annet i en bok for omreisende selgere i Sveits og Tyskland. Problemet ble formulert matematisk på 1800-tallet av W.R Hamilton og Thomas Kirkman \cite{c}. 
				Den generelle formen av TSP ble siden studert av matematikere på 1930-tallet, mest kjent av Karl Menger \cite{einstein}. dsqwfdcewfewfewfew \cite{a}

\newpage
\bibliographystyle{apacite}

\bibliography{references}

\end{document}



% set font \fontfamily{qcr}\selectfont