\documentclass[notitlepage]{article}
\usepackage[left=1in, right=1in, top=1in, bottom=1in]{geometry}
\usepackage{fancyhdr}

\usepackage{titling}
\usepackage{lipsum}

\pretitle{\begin{center}\Huge\bfseries}
\posttitle{\par\end{center}\vskip 0.5em}
\preauthor{\begin{center}\Large\ttfamily}
\postauthor{\end{center}}
\predate{\par\large\centering}
\postdate{\par}
\pagestyle{fancy}
\fancyfoot{}
\fancyfoot[R]{\thepage}
\renewcommand{\headrulewidth}{0pt}
\lhead{DA-OPT 3900: PROSJEKTRAPPORT FOR OBLIGATORISK OPPGAVE 1}
\rhead{}


\title{DA-OPT 3900:\\Prosjektrapport for obligatorisk oppgave 1}
\author{Amer Sisic, Thor-Stian Follstad og Martin Gran}
\date{\today}
\renewcommand*\contentsname{Innholdsfortegnelse}


\begin{document}
% setter starten på seksjoner
\setcounter{section}{1}

\maketitle
\thispagestyle{myheadings}
\vspace{\fill}
\begin{abstract}
This project report describes the implementation of together four different algorithms of the "Travelling salesman problem" (TSP).
First a symetrical complete graph is implemented with random lengths of 1-10 between a random number of cities. Then a route is 
chosen on random and the cost is calculated. The next solution is to choose routes at random and updating the best route by the 
lowest cost. Third a greedy method is implemented. In the implementation the next city in the route is chosen by selecting
the path with the lowest value. Fourth a greedy itterative method is chosen. The method works on a complete route, then switches
cities at random and updates the route if the total cost is improved. The fourth method works on routes generated by the previous
methods. Last the different solutions are run 100 times at the same graph. The best result and the mean result of the methods are then compared.  
\\ \\ \\ \\ \\ \\ \\ \\ \\
\end{abstract}
\vspace{\fill}
\pagebreak
\thispagestyle{empty}%
\tableofcontents
\pagebreak
\addcontentsline{toc}{section}{1 Innledning}
\indent Denne prosjektrapporten beskriver fire forskjellige implementasjoner av algoritmer for å løse "Travelling Salesman problem" (TSP).

\pagebreak
%\section*{First bit}
%If you want to ramp your text straight onto the title page, start the text at 
%something that does not cause a page break, like a section.  Here's a handy 
%place to introduce some of your woofy conventions, like quotes in equations.

\section{Teoretisk bakgrunn}
\subsection{Travelling Salesman problemet}
A new chapter starts a new page.  
\end{document}
% set font \fontfamily{qcr}\selectfont